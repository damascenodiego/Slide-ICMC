% Template customized by Diego Damasceno (damascenodiego@usp.br)
% Based on Rouben Rostamian's beamer template
% URL: http://userpages.umbc.edu/~rostamia/beamer/

\documentclass[xcolor=pdftex,dvipsnames,aspectratio=169]{beamer}
%\usetheme{Boadilla}
\usetheme{Malmoe}
%\usetheme{boxes}
%\usetheme{default}

\usepackage[brazil]{babel}   
\usepackage[latin1]{inputenc}

% \usepackage{enumitem}
\usepackage{caption}
\usepackage{transparent}

\setbeamertemplate{caption}[numbered]

%see more color names on http://www.math.umbc.edu/~rouben/beamer/quickstart-Z-H-24.html
\definecolor{uspblue}{RGB}{17,96,165} 
\usecolortheme[named=uspblue]{structure} 

%Use RGB colours instead of names
%\usecolortheme[RGB={205,173,0}]{structure} 

%download from http://www.math.umbc.edu/~rouben/beamer/beamer-umbc.tar.gz
\useoutertheme{umbcfootline} 


%Not needed if the umbcfootline outertheme is used:
\setbeamertemplate{navigation symbols}{}

%Custom footline. More options (also without the ``short'') part:
%\insertshortauthor, \insertshortinstitute, \insertshorttitle, \insertshortsubtitle, \insertshortdate, \insertframenumber{}, \inserttotalframenumber
\setfootline{\insertshortauthor, \insertshortinstitute \hfill \inserttitle 
    \hfill slide \insertframenumber{} /\inserttotalframenumber} 
 


%Set the background image.
\setbeamertemplate{background 
canvas}
{\transparent{0.25}\includegraphics[width=\paperwidth,height=\paperheight]{logos/usp_bg.png}}


\title[short title]{Long title}
%\subtitle[short subtitle]{with OU/Comlab logos}
%\author[Diego Damasceno]{Carlos Diego Nascimento Damasceno}
\author[short name author]{ long name author} \institute[short institute]{long institute  \href{mailto:mail@mail.com}{mail (at) mail (dot) com }} 


\begin{document}
\begin{frame}
	\maketitle
\end{frame}

\begin{frame}{Section}
\begin{itemize}
\item item\ldots
\begin{itemize}
	\item item\ldots
\end{itemize}
\end{itemize}

\begin{theorem}
It is impossible to separate any power higher than the second into two like powers.
\end{theorem}

\end{frame}

\begin{frame}{Two Columns}
    \begin{columns}[c] % the "c" option specifies center vertical alignment
    \begin{column}[c]{.5\textwidth} % column designated by a command
    	\begin{itemize}
  		\item testing
  		\item whatever\ldots
		\end{itemize}
    \end{column}
    \begin{column}[c]{.5\textwidth} % column designated by a command
		\begin{figure}
		\centering
		\captionsetup{justification=centering}
		\includegraphics[width=.5\textwidth]{logos/cnpq_logo.png}
		\caption{my caption \cite{aReference}}
		\end{figure}
    \end{column}
    \end{columns}

\end{frame}


%referencias
\begin{frame}{References}
\bibliographystyle{sbc}
\bibliography{refs}
\end{frame}

\begin{frame}
	%\maketitle
	\begin{center}
	\Huge{Thank you!}
	\\~	\\~
	\begin{figure}
	\captionsetup{justification=centering}
	\includegraphics[height=30pt]{logos/garland_logo_mini.png}
	~~~\includegraphics[height=30pt]{logos/icmc_logo_mini.jpg}
	~~~\includegraphics[height=30pt]{logos/usp-logo-png_mini.png}
	~~~\includegraphics[height=30pt]{logos/cnpq_logo.png}
	\end{figure}
	
	\end{center}
\end{frame}


\end{document}